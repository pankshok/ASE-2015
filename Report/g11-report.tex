\documentclass[a4paper,12pt,titlepage,fullpage]{article} %page properties
\usepackage[headings]{fullpage} %for fullpage with headings
\usepackage{fancyhdr} %for headings
\usepackage{graphicx} %for label
\usepackage{tabularx} %for fullpage tables
\usepackage{longtable} %for tables on multiple pages
\usepackage{hyperref}


% defining headings
\pagestyle{fancy}
\fancyhead{}
\fancyhead[LE,RO]{ASE 2015-2016 \hfill PROJECT REPORT}

\begin{document}
\begin{titlepage}

%defining university
\begin{center}
	POLITECNICO DI MILANO --- COMO CAMPUS\\
	\vspace{10pt}
	\includegraphics[scale=0.1]{logo-polimi.png}\\
	\vspace{10pt}
	ADVANCED SOFTWARE ENGINEERING 2015-2016\\
	prof. Marco Brambilla
	\line(2,0){500}
\end{center}

\vspace{60pt}

%defining work type
\begin{center}
	{\Huge \textbf{Group 11}}\\
	{\Huge \textbf{Project report}}\\
\end{center}

\vspace{60pt}

%defining project title

\begin{center}
	{\large Project title}
\end{center}
\begin{tabularx}{\textwidth}{|X|}
	\hline
	A2 --- Analysis of models for identifying recurring patterns\\
	\hline
\end{tabularx}

\vspace{20pt}

%defining project repo
\begin{center}
	{\large Project repository}
\end{center}
\begin{tabularx}{\textwidth}{|X|}
	\hline
	\href{https://github.com/attillax/ASE-2015}{Click}\\
	\hline
\end{tabularx}

\vspace{20pt}

%defining group members
\begin{center}
	{\large Group members}
\end{center}
\begin{tabularx}{\textwidth}{|l|l|l|X|}
	\hline
	ID & Surname & Name & e-mail \\
	\hline
	10460625 & Golubeva & Svetlana & svetlana.golubeva@mail.polimi.it \\
	\hline
	 & Zamani & Azadeh & azadeh.zamani@mail.polimi.it \\
	\hline
	 &  & & \\
	\hline
	 &  & & \\
	\hline
\end{tabularx}

\vspace{\fill}
\begin{center}
	\line(2,0){500}
\end{center}

\end{titlepage}

% % % % % % % % % % % % % % % % % % % % % % % % % % % % % % % % %
\tableofcontents

% % % % % % % % % % % % % % % % % % % % % % % % % % % % % % % % %
\newpage
\listoftables

\listoffigures

% % % % % % % % % % % % % % % % % % % % % % % % % % % % % % % % %
\newpage
\section{Task} 

A2 --- Analysis of models for identifying recurring patterns.\\

\begin{tabular}{|c|c|c|}
	\hline
	Students & Interactions & Marks \\
	\hline
	from 3 to 5 & every 2 weeks & from 5 to 10 \\
	\hline
\end{tabular}

\vspace{15pt}

The project consists in:

\begin{itemize}
	\item studying the existing research literature in the field of pattern mining and identification in models
	\item selecting and possibly advancing one technique, and implementing an analyzer that applies such model identification approach to ecore metamodels
	\item applying the analysis to a set of metamodels available from past exercises and exams
	\item showing and discussing the results
\end{itemize}


% % % % % % % % % % % % % % % % % % % % % % % % % % % % % % % % %
\newpage
\section{Technical Notes}
\subsection{Repository structure}

\subsection{Chronology}
\begin{longtable}{|c|p{13cm}|}
	\hline
	Date & Content \\
	\hline
	2015-12-16 & Group forming \\
	\hline
	2015-12-21 & Project assignment \\
	\hline
	2015-12-22 & Creation of the repository and the report template \\
	& Request for the clarifications about the project's task \& materials \\
	\hline
	& \\
	\hline
	2015-01-04 & Check1 \\
	\hline
	2015-01-18 & Check2 \\
	\hline
	2015-02-01 & Check3 \\
	\hline
	& \\
	\hline
	& \\
	\hline
	& \\
	\hline
	& \\
	\hline
	& \\
	\hline
	\caption{project's chronology.}
\end{longtable}

\subsection{List of tools}
\begin{table}[h]
	\begin{tabularx}{\textwidth}{|l|X|}
		\hline
		What & Which \\
		\hline
		OS & Linux \\
		\hline
		Lang & \TeX \\
		\hline
		IDE & \TeX-Studio, Emacs\\
		\hline
		& Terminal, make \\
		\hline
		& \\
		\hline
	\end{tabularx}
	\caption{list of tools.}
\end{table}


% % % % % % % % % % % % % % % % % % % % % % % % % % % % % % % % %
\end{document}
